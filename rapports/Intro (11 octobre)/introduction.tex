JavaCard est une technologie permettant d'exécuter des applets
Java sur des \textit{smartcards}, des cartes à puce (comme par exemple
des cartes bancaires ou des cartes SIM). Le but premier de JavaCard
est la sécurité : il est très simple d'utiliser des algorithmes de
chiffrement comme AES, Triple DES ou encore RSA.

\paragraph{}
Les applets JavaCard développées pour des \textit{smartcards} sont un
exemple concret du besoin de sécurité dans des projets
informatiques. En effet, une applet mal sécurisée peut potentiellement
être utilisée à des fins non prévues par le programmeur, ce qui peut
avoir des effets néfastes, par exemple dans le cas d'un système de
paiement. Notre projet a pour but de générer un automate représentant
une applet JavaCard de manière automatique, pour ensuite le comparer à
un automate de référence. Il sera donc possible de savoir les actions
qui font passer une applet d'un état à un autre et de remarquer
simplement les changements d'état non prévus.
