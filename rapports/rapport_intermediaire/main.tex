\documentclass[a4paper, 12pt]{report}

\usepackage[utf8]{inputenc}
\usepackage[T1]{fontenc}
\usepackage[french]{babel} 
\usepackage[top=35mm, bottom=35mm, left=25mm, right=25mm]{geometry}
\usepackage{geometry}
\usepackage{graphicx}
\usepackage{multirow}  
\usepackage{subfigure}
\usepackage{verbatim}
\usepackage{url}
\usepackage{algorithmic, algorithm} 
\usepackage{amsmath,amsfonts,amssymb}
\usepackage{lmodern}
\usepackage{microtype}
\usepackage{xcolor}
\usepackage{textcomp}
\usepackage{framed}
\usepackage{tikz}
\usepackage{tcolorbox}
\usepackage{etoolbox}
\usepackage{hyperref}
\usepackage{minted}

\usetikzlibrary{automata,arrows}

\definecolor{vert}{RGB}{56, 155, 34}
\definecolor{rouge}{RGB}{157, 29, 29}


\hypersetup{
    colorlinks,
    citecolor=black,
    filecolor=black,
    linkcolor=black,
    urlcolor=black,
}

\title{Analyse d'applets JavaCard et réalisation d'automates}
\author{Romain \bsc{Barrat}
  \and
  Simon \bsc{Garrelou}
  \and
  Clément \bsc{Jarrige}
  \and
  Marouan \bsc{Sami} \and
  ~\and
  Encadrant : Jean-Louis \bsc{Lanet}}

\date{13 janvier 2017}


\setcounter{secnumdepth}{3}
\usepackage{fancyhdr}
\pagestyle{fancy}
 \lhead{\leftmark}
 \rhead{}

 
%création du glossaire
%Création des doubles entrées acronyme-glossaire
\usepackage{xparse}

\usepackage[acronym]{glossaries}

\makeglossaries

\DeclareDocumentCommand{\newdualentry}{ O{} O{} m m m m } {
  \newglossaryentry{gls-#3}{name={#5},text={#5\glsadd{#3}},
    description={#6},#1
  }
  \newacronym[see={[Glossary:]{gls-#3}},#2]{#3}{#4}{#5\glsadd{gls-#3}}
}

%exemple d'entrée double :
%\newdualentry{OWD} % label
%  {OWD}            % abbreviation
%  {One-Way Delay}  % long form
%  {The time a packet uses through a network from one host to another}



\newdualentry{VM}
    {VM}
    {machine virtuelle}
    { ou Virtual Machine en anglais est une machine (un ordinateur par exemple) simulée par un autre équipement}
    
\newdualentry{JPF}
    {JPF}
    {Java Path Finder}
    { outil créé par le NASA pour effectuer la vérification du modéle d'une application Java}
    
\newdualentry{SMT}
    {SMT}
    {Satisfiability Modulo Theories}
    { la satisfiabilité modulo théories est un problème de décision pour des formules de la logique du premier ordre sans quantificateurs}
    
\newdualentry{APDU}
    {APDU}
    {Application Protocol Data Unit}
    { ou Application Protocol Data Unit est un message électronique permettant la communication avec une carte à puce (Normalisé dans l'ISO 7816)}

\newglossaryentry{GNU-Linux}
{
        name=GNU-Linux,
        description={ est le nom associé à un système d'exploitation alliant le noyau Linux, développé par Linus Torvalds, et les outils du projet GNU initié par Richard M. Stallman}
}
\newglossaryentry{bytecode}
{
	name=bytecode,
	description={ est un code intermédiaire entre les instructions machines et le code source}
}
\newglossaryentry{microsoft-z3}
{
        name=Microsoft-Z3,
        description={ Z3 est un prouveur de théorème créé par Microsoft Research. Il est autorisé en vertu de la licence MIT}
}

\newglossaryentry{windows}
{
        name=Windows,
        description={ est le principal système d'exploitation développé par Microsoft}
}

\newacronym{bir}{BIR}{Bandera Intermediate Representation}
\newacronym{birc}{BIRC}{Bandera Intermediate Representation Constructor}
\newacronym{babs}{BABS}{Bandera Abstraction-Based Specializer}

\newdualentry{SSA}
	{SSA}
	{Static single assignment}
	{ ou Affectation unique statique, est une représentation intermédiaire de source code qui n'autorise les variable à être affecté qu'une seule fois}



\begin{document}
	\begin{titlepage}
\begin{center}

    \includegraphics[width=0.5\textwidth]{images/logo-ul.png} ~\\[1cm]
    \textsc{\LARGE Faculté des Sciences et Techniques} \\[0.5cm]
    \textsc{\Large Département Informatique} \\[1.5cm]

    % Titre
    \vspace*{3,5cm}

    \hrule height 0.05cm ~\\[0.4cm]
    {\huge \bfseries Projet (Analyse et développement logiciel)} \\[0.5cm]
    {\LARGE \bfseries Analyse d'applets JavaCard et réalisation d'automates} \\[0.5cm]
    \hrule height 0.05cm ~\\[0.6cm]

    {\large Romain \bsc{Barrat} ---
	Simon \bsc{Garrelou} ---
	Clément \bsc{Jarrige} ---
	Marouan \bsc{Sami} }

    ~\\[2cm]


    % Footer
    \vfill
    Encadrant : 	\bsc{Lanet} Jean-Louis \\
	Responsable Module : \bsc{Crespin} Benoît\\
    {\large  2016 - 2017 \\}

\end{center}
\end{titlepage}

	\newpage
	~
	\thispagestyle{empty}
	
	\setcounter{page}{1}
\chapter*{Introduction}

\paragraph{}
Au cours de notre premier semestre de master Informatique, nous avons eu à choisir un sujet de projet qui sera à traiter tout au long de l'année.
Étant quatre étudiants nous dirigeant vers le master \bsc{Cryptis}, nous avons choisi de prendre un sujet en lien avec la sécurité informatique, et notamment avec la sécurité des cartes à puces fonctionnant sous JavaCard.

\paragraph{Choix du projet}
Sous l'impulsion de Monsieur Jean-Louis \bsc{Lanet}, nous avons donc choisi le projet intitulé ``Analyse d'applets JavaCard et réalisation d'automates''.

\paragraph{Objectif du projet}
L'objectif du projet est d'être capable de réaliser une application permettant de générer, à partir du code source d'une applet JavaCard, un automate ayant le même fonctionnement que l'applet en question. Cette opération aura pour objectif de pouvoir tester, en simulant un fonctionnement réel via l'automate, l'applet jusque dans ces détails et ainsi révéler d'éventuelles failles de sécurité.

\paragraph{Déroulement du projet}
Dans ce rapport préliminaire, nous allons vous présenter comment et avec quels outils nous comptons, au semestre prochain, atteindre notre objectif.

Nous commencerons par présenter quelle(s) technologie(s) déjà existante(s) réalise(nt) ou approche(nt) des travaux en lien avec notre objectif. Puis nous présenterons un outil qui permet d'utiliser la technique des exécutions concoliques dans le but de tester des applications, viendra ensuite les moyens de réalisation d'un automate généré grâce aux tests et enfin le stockage des graphes liés à l'automate généré.

Une dernière partie consistera à présenter le travail à réaliser lors du prochain semestre.
	\newpage
	\tableofcontents

	\chapter{Objectif du Projet}
\paragraph{}
Le but de ce projet est de pouvoir extraire un automate à partir du code source Java d’une applet JavaCard et le comparer à un automate de référence. Dans cette partie, afin d'expliquer au mieux les enjeux et les besoins nous commenceront par expliquer l'intérêt de l'extraction d'automate pour une application sous JavaCard, puis nous donnerons un court exemple de ce qui sera demandé à l'application.

\section{Intérêt de l'extraction d'automates à partir de code source}



\paragraph{Un objectif de sécurité}En effet, une application pour une carte à puce est souvent liée à un domaine en rapport à la sécurité (passeport de sécurité par exemple), à la banque (carte bancaire) ou bien encore à la téléphonie (carte SIM). Tout ces domaines d'application sont liés intimement à des données pouvant être très sensibles ou à garder à la seule connaissance de l'utilisateur légitime.

Exporter une application sous forme de graphe permet de vérifier la présence éventuelle de faille de sécurité.

\paragraph{Une autre utilisation possible}Comme nous l'avons dit ci-dessus, un tel outil peut être employé à la sécurisation de code et ainsi aider le développeur à générer une application la plus solide possible. Un autre aspect peut être celui de l'attaque. Il est tout à fait raisonné de penser qu'entre les mains d'experts, ce genre d'outil peut aider les chercheurs à trouver de nouvelles failles et ainsi les révéler pour les corriger.


\section{Mise en pratique sur une applet simple}

\paragraph{}
Tout d’abord il est nécessaire de décrire les différents états de l’applet ainsi que les transitions entre les états qui correspondent à la réception d’une commande \gls{APDU}. Nous devons repérer dans le code les réceptions de commande \gls{APDU}, déterminer les valeurs qu’elles peuvent prendre et créer l’arbre des contraintes de l’applet.
\paragraph{}
Ensuite nous pouvons construire un automate montrant les états et les commandes \gls{APDU} à envoyer pour passer de l’un à l’autre. Il sera alors possible de le comparer à l’automate de spécification de l’applet.

\paragraph{}
Le programme suivant est la partie métier d’un applet simplifiée de gestion bancaire. Cet applet propose deux fonctionnalités : consulter son solde avec le code 0x34 ou retirer de l’argent avec le code 0x38. Une protection basique est en place pour empêcher l’utilisateur de retirer plus d’argent que son compte n’en possède.

\textit{Note~:} Le tableau \verb|buffer| contient l'\gls{APDU}.

\begin{minted}[fontsize=\footnotesize]{java}
public void process(APDU apdu) throws ISOException {
  byte[] buffer = apdu.getBuffer();
      
  if(buffer[ISO7816.OFFSET_CLA] != (byte) 0x80)
    ISOException.throwIt(ISO7816.SW_CLA_NOT_SUPPORTED);
      
  short octetsLus;
      
  switch(buffer[ISO7816.OFFSET_INS]){
  case (byte) 0x34:
      octetsLus = apdu.setIncomingAndReceive();
      if(octetsLus != (short) 0)
	ISOException.throwIt(ISO7816.SW_WRONG_LENGTH);
      if(buffer[ISO7816.OFFSET_P1] != (byte) 0x00 
	  && buffer[ISO7816.OFFSET_P2] != (byte) 0x00 )
	ISOException.throwIt(ISO7816.SW_INCORRECT_P1P2);
      Util.setShort(buffer, (short) 0, balance);
      apdu.setOutgoingAndSend((short)0 , (short)2);
      return;
    case (byte) 0x38:
      octetsLus = apdu.setIncomingAndReceive();
      if(octetsLus != (short)2)
	ISOException.throwIt(ISO7816.SW_WRONG_LENGTH);
      if(buffer[ISO7816.OFFSET_P1] != (byte)0x00 
	  && buffer[ISO7816.OFFSET_P2] != (byte)0x00)
	ISOException.throwIt(ISO7816.SW_INCORRECT_P1P2);
      short delMont = Util.getShort(buffer, (short)5);
      if(balance < delMont)
	ISOException.throwIt(ISO7816.SW_COMMAND_NOT_ALLOWED);
      balance -= delMont;
      return;
  default:
      ISOException.throwIt(ISO7816.SW_INS_NOT_SUPPORTED);
  }

}
\end{minted}

Une fois l'application analysée par le programme d'analyse concolique, on souhaite obtenir un graphe de la forme : 

~

\begin{tikzpicture}[very thick,top color=white,bottom color=gray]
\node[anchor=west,draw] (A) at (0,0) {buffer[0] != 0x80 \textbf{throw ISOException}};
\node[anchor=west,draw] (B) at (1,-1) {buffer[1] == 0x34};
\node[anchor=west,draw] (C) at (2,-2) {APDU.setIncomingAndReceive() == 0 \textbf{throw ISOException}};
\node[anchor=west,draw] (D) at (3,-3) {buffer[2] != 0 and buffer[3] != 0 \textbf{throw ISOException}};
\node[anchor=west,draw] (E) at (4,-4) {Fin};
\node[anchor=west,draw] (F) at (1.1,-5) {buffer[1] == 0x38};
\node[anchor=west,draw] (G) at (2,-6) {APDU.setIncomingAndReceive() == 0 \textbf{throw ISOException}};
\node[anchor=west,draw] (H) at (3,-7) {buffer[2] != 0 and buffer[3] != 0 \textbf{throw ISOException}};
\node[anchor=west,draw] (I) at (4,-8) {Util.getShort(buffer, (short)5) < balance \textbf{throw ISOException}};
\node[anchor=west,draw] (J) at (5,-9) {Fin};
\node[anchor=west,draw] (K) at (1.1,-10) {default \textbf{throw ISOException}};


\draw (A.west) |- (B.west);
\draw (B.west) |- (C.west);
\draw (C.west) |- (D.west);
\draw (D.west) |- (E.west);
\draw (B.west) |- (F.west);
\draw (F.west) |- (G.west);
\draw (G.west) |- (H.west);
\draw (H.west) |- (I.west);
\draw (I.west) |- (J.west);
\draw (B.west) |- (K.west);

\end{tikzpicture}

	
        \chapter{Bandera~: génération de modèles à partir de code source Java}

\paragraph{}
Bandera~\cite{bandera1} est un logiciel permettant d'analyser un
projet Java et d'en extraire un modèle utilisable par un vérificateur
de code. Il a été développé suite à la collaboration de chercheurs de
l'Université d'Hawaï et de l'Université d'État du Kansas.

\paragraph{}
Ce logiciel n'est plus maintenu depuis 2005, néanmoins il peut être
intéressant de voir les décisions prises par ses développeurs pour
réaliser ce programme, en effet notre projet possède certaines
similarités~: nous cherchons aussi à analyser du code source Java afin
d'en extraire un automate.

\paragraph{}
Nous allons donc étudier son fonctionnement afin de comprendre comment
il fonctionne et possiblement s'en inspirer pour notre propre projet.

\section{Postulat de base}

\paragraph{}
Les développeurs de Bandera partent d'un constat~: des programmes de
vérification de modèle existent déjà, ils permettent de comparer un
modèle représentant un programme à un modèle de
référence. Malheureusement, créer ce modèle à la main est un travail
long, fastidieux et sujet à des erreurs humaines. De plus, chaque
vérificateur attend le modèle dans son propre langage ou format de
fichier, ce qui signifie que pour vérifier un programme avec plusieurs
vérificateurs, il faut effectuer ce travail plusieurs fois.

\paragraph{}
Leur but était donc d'écrire un programme permettant d'éliminer
l'humain de l'équation en analysant le code source et en générant des
fichiers lisibles par les vérificateurs.

\section{Fonctionnement de Bandera}
\subsection{Fonctionnement général}

\paragraph{}
Bandera est composé de quatre composants principaux qui seront pour la
plupart détaillés dans ce chapitre~:

\begin{description}
\item[Découpage du code (Section~\ref{sec:bandera_slicing})] Le découpage de code permet de supprimer les
  données et les points de contrôle qui ne sont pas nécessaires à la
  vérification d'une certaine propriété.
\item[Abstraction des variables (Section~\ref{sec:bandera_abstraction})] Bandera peut rendre certaines
  variables abstraites et les remplacer par des opérations possibles
  sur cette variable.
\item[Transformation du code source (Section~\ref{sec:bandera_source})] Afin de générer un fichier
  compatible avec les vérificateurs de modèle, Bandera passe par une
  représentation propre à ses outils.
\item[Interface graphique] Bandera est utilisable grâce à une
  interface graphique, mais celle-ci ne sera pas détaillée car elle
  n'est pas nécessaire à la compréhension de son fonctionnement.
\end{description}

\subsection{Découpage (\textit{slicing}) du programme}
\label{sec:bandera_slicing}

\paragraph{}
Dans le but de ne garder que les instructions intéressantes, Bandera
effectue un découpage des instructions. Pour un programme $P$, des
opérations $s_i$ sont extraites et regroupées dans $C$.

$$C = \{s_1, s_2, \ldots, s_k\}$$

\paragraph{}
Le découpage du programme va réduire $P$ afin de ne laisser que les
instructions nécessaires au bon fonctionnement des opérations
contenues dans $C$, et supprimer les autres.

\paragraph{}
Bandera se sert de cette méthode pour vérifier qu'un programme $P$
adhère à une spécification $\Phi$. Le logiciel supprime les
instructions de $P$ qui n'influent pas la satisfaction de $\Phi$. Si
$\Phi$ est juste pour la version réduite de $P$, alors $\Phi$ est
juste pour la version complète de $P$.

\paragraph{}
Découper un programme permet aussi de simplifier le travail du
vérificateur de modèle en réduisant le nombre d'instructions à
vérifier.

\subsection{Abstraction des variables}
\label{sec:bandera_abstraction}

\paragraph{}
Bandera est capable de considérer des variables et méthodes comme
\textit{abstraites}, ce qui signifie que leur valeur concrète n'est
pas spécifiée. Cette abstraction permet de réduire la taille du modèle
et de simplifier le travail du vérifieur.

\paragraph{}
Cette abstraction est effectuée par le \gls{babs}, un composant
spécialisé de Bandera. Lorsqu'une variable est définie abstraite, elle
n'est plus représentée par sa valeur mais par certaines
propriétées. Ces propriétés dépendent de la spécification, par exemple
si une variable de type \verb|ArrayList| est abstraite et qu'on ne
fait que vérifier si une valeur est contenue dedans, elle peut être
remplacée par l'abstraction
$\{ ElementPresentDansListe, ElementAbsentDansListe \}$. Dans ce cas
précis, le \gls{babs} remplacera toutes les actions effectuées sur
l'\verb|ArrayList| par des versions abstraites manipulant des symboles
représentant des deux valeurs possibles, $ElementPresentDansListe$ et
$ElementAbsentDansListe$.

\paragraph{}
Si une opération impossible à représenter par ces deux valeurs est
demandée, par exemple \verb|ArrayList.size()|, alors l'abstraction
renvoie une valeur spéciale notée $\top$.

\subsection{Transformation du code source}
\label{sec:bandera_source}

\paragraph{}
Afin de transformer du code Java en un langage utilisable par des
vérificateurs, Bandera passe par un certain nombre de langages
intermédiaires. Il commence par traduire le programme en Jimple, un
langage utilisé par le framework Soot et établit des correspondances
entre le code Java et le code Soot. Ceci permet au logiciel de
retrouver le n\oe{}ud Java correspondant à n'importe quel n\oe{}ud
dans Jimple.

\paragraph{}
L'arrière-plan (\textit{backend}) de Bandera s'occupe de traduire le
code Jimple en \gls{bir}, un langage bas niveau qui abstrait les
concepts communs à de nombreux logiciels de vérification de modèle. Le
but ce ce langage intermédiaire est de pouvoir ensuite exporter le
\gls{bir} dans différents langages spécialisés pouvant être utilisés
par les vérificateurs de modèles. La traduction est faite par
\gls{birc}.

\paragraph{}
Ce fonctionnement est schématisé dans la figure~\ref{fig:bir_jimple}.

\begin{figure}[H]
  \centering
  \includegraphics[scale=0.5]{images/bandera_bir_jimple.png}
  \caption{\label{fig:bir_jimple} Transformation du code source en
    langages intermédiaires}
\end{figure}

\section{Interêt et limites}

\paragraph{}
Bandera est un outil très puissant et complexe qui gère même l'analyse
de threads Java. Les développeurs de Bandera ont choisi de ne pas
utiliser JDart (présenté dans la section suivante) pour effectuer
l'analyse afin de pouvoir modifier plus simplement le moteur d'analyse
lorsque de nouvelles avancées seront disponibles. JDart est pourtant
un outil très puissant et toujours maintenu permettant d'analyser des
applications Java.

\paragraph{}
Sachant que l'API JavaCard ne gère pas les threads et que nous
utiliseront JDart pour effectuer l'analyse, il sera plus simple pour
nous de travailler sur le code source des applets afin d'en extraire
un modèle.

\paragraph{}
Il apparaît que Bandera est un projet très intéressant pour la
vérification formelle de code. Cependant, la plupart des
fonctionnalités qui nous intéressent --- l'analyse du code, des
changements des variables, etc. --- sont aussi réalisables en
utilisant l'outil JDart, qui sera présenté dans la prochaine
section. Celui-ci est encore maintenu et est plus orienté vers les
développeurs, il sera donc plus simple à intégrer à notre projet.


        \chapter{JDart et exécution concolique}
  \paragraph{Introduction}
    L'idée générale est de rendre la tâche de tester des Applets Java Card moins penible et plus effective que possible.
    Pour celà nous optons pour l'utilisation de l'execution concolique comme une technique d'analyse ce qui permet de rendre 
    le system à tester moins obscue et plus predictible surtout quand on est face à des systemes complexe
    et qui nécessite des méthodes plus avancées qu'un simple teste unitaire.
  \section{Exploration des chemins et l'execution concolique}
    \paragraph{Java Path Finder: Exploration des chemins}
  \section{La VM JAVA, La VM du JPF et la JCVM}
  \section{Tester des applets JavaCard}
  \paragraph{Conclusion}
  Java Path Finder connu comme le couteau Suisse de la verification Java,
  est en effet un des outis de teste les plus evolués pour les applications Java,
  grâce à son extensibilité et à son abilité de supporter et interger de nouvelle extensions.
  Cependant, la majorité des extensions ne sont pas destinés à être executer sur des systemes Java Card.
  

        \chapter{Génération d'un automate}

\paragraph{}
Dans cette partie nous allons présenter l'outil qui nous permettra de créer 
l'automate d'une applet afin de pouvoir le comparer à l'automate fourni dans les 
spécifications de l'application. Dans nos recherche nous avons trouvé l'outil 
Soot \cite{Soot}.

	\section{Présentation de Soot}

	\paragraph{}
	À l'origine Soot est un framework qui à pour but d'optimiser du code. 
Soot a été produit par le groupe de recherche Sable de l'université de McGill. 
Concrètement ce framework permet de :

	\begin{itemize}
		\item analyser points par points
		\item créer des graphes de séquence
		\item créer des graphes de flux de controle
		\item instrumeter du code
	\end{itemize}

	\paragraph{}
	Soot propose plusieurs représentations intermédiaires du code. Chaque 
représentation intermédiaire correspond à un niveau d'abstraction.

	\paragraph{Baf} est une représentation simplifiée du bytecode Java. Il 
permet de manipuler du bytecode plus simplement.
	\paragraph{Jimple} est la représentation principale de Soot. C'est une 
représentation à trois addresses, c'est-à-dire qu'une instruction peut contenir 
trois variables. Par exemple l'instruction suivante : 
	$$ a = (b + 1) * c - d$$
	Devient : 
	$$ \$i_0 = b + 1$$
	$$ \$i_1 = \$i_0 * c$$
	$$ a = \$i_1 - d$$
	où \$i0 et \$i1 sont des variables intermédiaires.


	Le jeux d'instructions est aussi réduit (15 instructions contre plus de 
200 en bytecode).
	\paragraph{Shimple} est une représentation de type \gls{SSA}, c'est une 
représentation qui interdit la réaffectation d'une variable, des versions de 
variables sont utilisées pour remplir ce critère. Par exemple le code suivant :
	$$ y = 1 $$
	$$ y = 2 $$
	$$ x = y $$
	Devient :
	$$ y_1 = 1 $$
	$$ y_2 = 2 $$
	$$ x_1 = y_2 $$ 
	Cela permet de voir que la première instruction est inutile. Shimple 
permet donc d'éliminer du code mort, de contrôler la propagation de constante...

	\paragraph{Grimp} est une représentation basé sur Jimple, elle 
n'utilise pas de variable intermédiaire ce qui la rapproche plus du langage 
Java. Elle est utiliée pour de la décomplilation ou pour de l'inspection de 
code.

\section{}

\paragraph{}

        \chapter{Représentation des graphes}
\paragraph{}

Comme nous l'avons vu auparavent, il est nécessaire de pouvoir stocker les automates générés dans la partie précédente. Pour cela, les graphes nous ont été imposés. Dans cette partie, nous vous présenteront donc les moyens que nous avons à disposition pour réaliser le stockage de graphes et nous les compareront afin d'en selectionner le meilleur.

\section{Différents format de représentation}
  \paragraph{}
  Il existe de nombreux formats de représentation de graphes tels que le JSON, le CSV ou bien encore l'XML.
  
  Dans cette première partie, nous allons nous concentrer sur le choix du format que nous utiliserons pour stocker et réutiliser nos graphes après leur création dans l'application.
  Nous allons pour cela, passer par un tableau comparatif des diverses solutions.
  

  \paragraph{Liste des formats de représentation de graphes}
  La liste ci-dessous contient tous les formats de représentation de graphes à comparer : 
  \begin{itemize}
   \item CSV : est un type de fichier très répendu, il représente les informations sous formes de tableaux et est facilement lisible via la plupart des programmes. 
   \item XML : est surement le type de fichier le plus répendu, stockant les donées sous forme de balises, il est très utilisé pour toutes les applications webs et orientées objets. 
   \item JSON : le petit nouveau, il est le type de fichier favoris de tous les langages orientés objets car il permet un stockage très simple de tous les types de données, des plus simples aux plus complexes. 
   \item svg : basé sur le XML, ce format est conçu pour supporter les images vectorielles.
   \item bmp : est un format d'image propriétaire lancé par Microsoft, il est assez répendu.
   \item jpg : est un autre format d'image, plus répendu que le bmp.
  \end{itemize}

  
  \paragraph{Liste des éléments déterminants le choix}
  Avant de pouvoir comparer les différents types de stockage des graphes, il nous faut déterminer une grille de critères nous permettant de classer les différentes solutions en lice.
  
  La solution retenue devra donc respecter les points suivants : 
  \begin{itemize}
   \item utiliser un format libre de données.
   \item assurer la portabilité des information en étant lisible par le plus grand nombre d'applications.
   \item être simple de lecture et de traitement, tirer facilement un graphe sous forme graphique des données stockées.
   \item facilement modifiable : si on souhaite rajouter des éléments au graphe
   \item supporter le langage objet nativement, car l'automate généré sera représenté sous forme d'objet et les graphes le seront également.
  \end{itemize}
  
  \begin{table}[!h]{Tableau comparatif des formats de fichiers}
  \centering
    \begin{tabular}{p{3.5cm} p{0.50cm} p{0.5cm} p{0.50cm} p{0.5cm} p{0.50cm}}

	& \makebox[0cm][l]{\rotatebox{45}{ Format libre }} &
	\makebox[0cm][l]{\rotatebox{45}{ Inter langages }} &
	\makebox[0cm][l]{\rotatebox{45}{ Conversion en image simple}} &
	\makebox[0cm][l]{\rotatebox{45}{ Evolution simple }} &
	\makebox[0cm][l]{\rotatebox{45}{ Support objet natif }}\\
	    
    \end{tabular}
    
    \begin{tabular}{|p{3.5cm} | p{0.50cm} | p{0.5cm} | p{0.50cm} | p{0.5cm} | p{0.50cm} |}
      
      \hline  
      \textbf{CSV} & x & x & x & x &  \\ \hline
      \textbf{XML} & x & x & x & x & x \\ \hline
      \textbf{JSON} & x & x & x & x & x \\ \hline
      \textbf{SVG} & x & x & x &  & x \\ \hline
      \textbf{BMP} & x & x & x &  &  \\ \hline
      \textbf{JPG} & x & x & x &  &  \\ 
      \hline
    \end{tabular}
  \end{table}
  
  \paragraph{Choix final}
  Après comparaison, nous voyons que deux formats peuvent correspondre à notre attente : l'XML et le JSON, nous allons donc nous baser sur ces résultats pour rechercher un outil nous permettant de représenter les graphes à l'aide d'un de ces formats.

  \section{Outils de représentation existants}
  
  \paragraph{}
  Après avoir déterminé les formats de fichiers les plus appropriés, il nous faut encore trouver un outil servant à représenter et dessiner les graphes ou, si nous n'en trouvons pas un qui conviendrait, se servir d'un outil pour base afin de le modifier en implémentant les fonctions recherchées.
  
  \paragraph{Fonctionnalités demandées et contraintes imposées}
  Pour répondre à tous nos besoins, la solution retenue devra implémenter les fonctionnalités suivantes : 
  \begin{itemize}
  	\item Proposer une sortie simple à comprendre pour l'utilisateur
  	\item Dessiner un graphe à partir d'un fichier source
  	\item 
  \end{itemize}
  
  \paragraph{}
  La solution retenue devra également répondre aux contraintes suivantes : 
  \begin{itemize}
  	\item Posséder une licence libre afin de pouvoir apporter d'éventuelles modifications au code source.
  	\item Être intégrable à notre application finale
  	\item 
  \end{itemize}

  \paragraph{Quelques solutions possibles et leur description}

  \begin{table}
  	\begin{tabular}{|c|p{2cm}|p{2cm}|p{8cm}|}
	  \hline
	  Nom & Langage & licence & description \\
	  \hline \hline
	  \textbf{Dracula \footnote{Dracula : https://www.graphdracula.net/}} & JavaScript & MIT & Dracula est une bibliothèque très simple à utiliser par les utilisateurs, des options de drag-and-drop y sont présentes. Côté développeur, les fonctions d'ajout de noeuds et de gestion globales des graphes ont été implantées ainsi que certains algorithmes commums (Dijksta, divers parcours...)\\
	  \hline
	  \textbf{MXGraph\footnote{MXGraph : https://www.jgraph.com/}}& JavaScript &  Apache 2.0 & MXGraph utilise un rendu SVG pour représenter les graphes, ce qui est un gros avantage, car si les graphes obtenus sont très complexes, il est toujours possible de zommer sur une zone sans perdre en lisibilité.\\
	  \hline
	  \textbf{JGraphx\footnote{JGraphx : https://github.com/jgraph/jgraphx}}&Java& BSD 3 & JGraphx est développé par les mêmes personnes que MWGraph, contrairement \\
	  \hline 
	  \textbf{GraphStream\footnote{GraphStream : http://graphstream-project.org/}}& Java & & \\
	  \hline
	  \textbf{Graph\footnote{Graph :  https://github.com/clue/graph}}& PHP & & \\
	  \hline
  	\end{tabular}

  \end{table}

  \section{Outils d'analyse existants}

  
  
        
        \chapter{Travail à accomplir}

\section{}

\begin{tikzpicture}[->,>=stealth',shorten >=1pt,auto,node distance=3cm,thick,main node/.style={circle,fill=blue!20,draw,minimum size=1cm,inner sep=0pt]}]

\node[draw=white, fill=couleur3, state, rectangle, rounded corners=3pt] (A){\textcolor{white}{\textbf{Code Source}}};
\node[draw=white, state, ellipse, fill=couleur2] (B) [right=4cm of A] {\textcolor{white}{\textbf{Soot}}};
\node[draw=white, state, ellipse, fill=couleur2] (C) [below of=A] {\textcolor{white}{\textbf{JPF}}};
\node[draw=white, fill=couleur3,state, rectangle, rounded corners=3pt] (D) [right of=C] {\textcolor{white}{\textbf{Bytecode}}};
\node[draw=white, fill=couleur3, state, rectangle, rounded corners=3pt] (E) [below of=C] {\textcolor{white}{\textbf{Sortie JDart}}};
\node[draw=white,fill=couleur4, state, rectangle] (F) [below=1cm of E] {\textcolor{white}{\textbf{Générateur d'automates}}};
\node[draw=white, fill=couleur3, state, rectangle, rounded corners=4pt] (G) [below =1cm of F] {\textcolor{white}{\textbf{Graphe}}};
\node[draw=white, state, ellipse, fill=couleur2, align=center] (H) [below =1cm 
of G] {\textcolor{white}{\textbf{Rendu graphique de graphe}} \\ 
\textcolor{white}{\textbf{(GraphStream)}}};

\node[draw=white, fill=couleur3, state, rectangle, rounded corners=3pt]  (I) [right =5cm of E] {\textcolor{white}{\textbf{Fichiers fournis ou générés}}};
\node[draw=white, state, ellipse, fill=couleur2]  (J) [below=0.5cm of I] {\textcolor{white}{\textbf{Outils à utiliser}}};
\node[draw=white,fill=couleur4, state, rectangle] (K) [below=0.5cm of J] {\textcolor{white}{\textbf{Outil à developper}}};

\draw[<->] (A)  --   node [ align=center, below] {Modifier
le code \\ source
(annotation)}  (B);

\draw[->] (C)  --   node [ align=center, above, rotate=90] {Exécution\\ concolique}  (E);

 \path[->]
  (A) edge node{} (C)
  (D) edge node{} (C)
  (F) edge node{} (G)
  (G) edge node{} (H)
  (E) edge node{} (F);
  

\end{tikzpicture}


        \chapter{Tâches futures et planification}

\shorthandoff{:!}
\begin{figure}[H]
  \centering
  \begin{ganttchart}[
    vgrid,hgrid,
    title height=1
    ]{2}{31}
    \gantttitle{Janvier}{6}  \gantttitle{Février}{6} \gantttitle{Mars}{6} \gantttitle{Avril}{6} \gantttitle{Mai}{6} \\
    \gantttitlelist{6,11,16,21,26,31}{1} \gantttitlelist{3,8,13,18,23,28}{1} \gantttitlelist{6,11,16,21,26,31}{1} \gantttitlelist{5,10,15,20,25,30}{1} \gantttitlelist{6,11,16,21,26,31}{1} \\
    \ganttgroup{Group 1}{2}{7} \\
    \ganttbar{Task 1}{2}{2} \\
    \ganttlinkedbar{Task 2}{3}{7} \ganttnewline
    \ganttmilestone{Milestone}{7} \ganttnewline
    \ganttbar{Final Task}{8}{12}
    \ganttlink{elem2}{elem3}
    \ganttlink{elem3}{elem4}
  \end{ganttchart}
\caption{Planning prévisionnel du second semestre}
\end{figure}

\section{Quelles tâches à effectuer au semestre prochain ?}
\section{Tableau de planification des tâches}
        
	\chapter*{Conclusion}

\paragraph{}
Tout au long de ce semestre, nous avons réalisé une étude de documents et de solutions possibles pour mener à bien la réalisation de notre projet lors du second semestre.

\paragraph{}
Après réalisation de ce travail, nous avons en notre possession les outils nécessaires à la bonne tenue du projet. Après avoir déterminé les différentes tâches de notre projet et avoir déouper ces dernières en sous tâches précises, nous avons également une bonne idée quant aux travaux que nous aurons à réaliser après le dépôt de ce rapport. 

\paragraph{}
L'étape de réflexion qu'a été la rédaction de ce rapport nous a permis de prendre du recul vis-à-vis de l'énoncé qui nous avait été donné au début de l'année. Ce recul nous permet de débuter la phase de réalisation de manière sereine, nous savons où nous allons et quelles étapes sont à parcourir. De plus, en cas d'imprévu, nous avons vu suffisamment de solutions et de techniques différentes pour pouvoir nous adapter si besoin.

\paragraph{}



    %Glossaire des Acronymes
    %\printglossary[type=\acronymtype]
    %Glossaire
    \glossarystyle{altlist}
    \printglossary[type=\acronymtype]
    \printglossary
    
    \clearpage 
    \bibliographystyle{plain-fr}
    \bibliography{bibliographie}
    \addcontentsline{toc}{section}{Références}


\end{document}