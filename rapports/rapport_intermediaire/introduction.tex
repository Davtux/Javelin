\setcounter{page}{1}
\chapter*{Introduction}

\paragraph{}
Au cours de notre premier semestre de master Informatique, nous avons eu à choisir un sujet de projet qui sera à traiter tout au long de l'année.
Etant quatre étudiants nous dirigeant vers le master \bsc{Cryptis}, nous avons choisi de prendre un sujet en lien avec la sécurité informatique, et notamment avec la sécurité des cartes à puces fonctionnant sous JavaCard.

\paragraph{Choix du projet}
Sous l'impulsion de Monsieur Jean-Louis \bsc{Lanet}, nous avons donc choisi le projet intitulé ``Analyse d'applets JavaCard et réalisation d'automates''.

\paragraph{Objectif du projet}
L'objectif du projet est d'être capable de réaliser une application permettant de générer, à partir du code source d'une applet JavaCard, un automate ayant le même fonctionnement que l'applet en question. Cette opération aura pour objectif de pouvoir tester, en simulant un fonctionnement réel via l'automate, l'applet jusque dans ces détails et ainsi révéler d'éventuelles failles de sécurité.

\paragraph{Déroulement du projet}
Dans ce rapport préliminaire, nous allons vous présenter comment et avec quels outils nous comptons, au semestre prochain, atteindre notre objectif.

Nous commencerons par présenter quelle(s) technologie(s) déjà existante(s) réalise(nt) ou approche(nt) des travaux en lien avec notre objectif. Puis nous présenterons un outil qui permet d'utiliser la technique des exécutions concoliques dans le but de tester des applications, viendra ensuite les moyens de réalisation d'un automate généré grâce aux tests et enfin le stockage des graphes liés à l'automate généré.

Une dernière partie consistera à présenter le travail à réaliser lors du prochain semestre.