\chapter{Bandera : génération de modèles à partir de code source Java}

Bandera est un logiciel permettant d'analyser un projet Java et d'en
extraire un modèle utilisable par un vérificateur de code. Il a été
développé suite à la collaboration de chercheurs de l'Université
d'Hawaï et de l'Université d'État du Kansas.

Ce logiciel n'est plus maintenu depuis 2005, néanmoins il peut être
intéressant de voir les décisions prises par ses développeurs car
notre projet semble être similaire à ce logiciel.

\section{Transformation du code source}

Afin de transformer du code Java en un langage utilisable par des
vérificateurs, Bandera passe par un certain nombre de langages
intermédiaires. Il commence par traduire le programme en Jimple, un
langage utilisé par le framework Soot, et établit des correspondances
entre le code Java et le code Soot. Ceci permet au logiciel de
retrouver le n\oe{}ud Java correspondant à n'importe quel n\oe{}ud
dans Jimple.

