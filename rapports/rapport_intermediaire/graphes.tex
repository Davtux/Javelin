\chapter{Représentation des graphes}
\paragraph{}
Comme nous l'avons vu auparavent, il est nécessaire de pouvoir stocker les automates générés dans la partie précédente. Pour cela, les graphes nous ont été imposés. Dans cette partie, nous vous présenteront donc les moyens que nous avons à disposition pour réaliser le stockage de graphes et nous les compareront afin d'en selectionner le meriileur.

\section{Différents format de représentation}
  \paragraph{}
  Il existe de nombreux formats de représentation de graphes tels que le JSON, le CSV ou bien encore l'XML.
  
  Dans cette première partie, nous allons nous concentrer sur le choix du format que nous utiliserons pour stocker et réutiliser nos graphes après leur création dans l'application.
  Nous allons pour cela, passer par un tableau comparatif des diverses solutions.
  
  \paragraph{Liste des éléments déterminants le choix}
  Avant de pouvoir comparer les différents types de stockage des graphes, il nous faut déterminer une grille de critères nous permettant de classer les différentes solutions en lice.
  
  La solution retenue devra donc respecter les points suivants : 
  \begin{itemize}
   \item 
  \end{itemize}

  \paragraph{Liste des formats de représentation de graphes}
  La liste ci-dessous contient tous les formats de représentation de graphes à comparer : 
  \begin{itemize}
   \item CSV : 
   \item XML : 
   \item JSON :
   \item 
  \end{itemize}

  
  \paragraph{Choix du format}
  
  
  \begin{table}
   \begin{tabular}{|c|c|c|}
    \hline
    Format & Avantages & Inconvéniants \\
    \hline
    \hline
    toot & toot & toot \\
    \hline
   \end{tabular}

  \end{table}
  

  
  