\chapter{Travail à accomplir}

\paragraph{}

\begin{tikzpicture}[->,>=stealth',shorten >=1pt,auto,node distance=3cm,thick,main node/.style={circle,fill=blue!20,draw,minimum size=1cm,inner sep=0pt]}]

\node[draw=white, fill=couleur3, state, rectangle, rounded corners=3pt] (A){\textcolor{white}{\textbf{Code Source}}};
\node[draw=white, state, ellipse, fill=couleur2] (B) [right=4cm of A] {\textcolor{white}{\textbf{Soot}}};
\node[draw=white, state, ellipse, fill=couleur2] (C) [below of=A] {\textcolor{white}{\textbf{JPF}}};
\node[draw=white, fill=couleur3,state, rectangle, rounded corners=3pt] (D) [right of=C] {\textcolor{white}{\textbf{Bytecode}}};
\node[draw=white, fill=couleur3, state, rectangle, rounded corners=3pt] (E) [below of=C] {\textcolor{white}{\textbf{Sortie JDart}}};
\node[draw=white,fill=couleur4, state, rectangle] (F) [below=1cm of E] {\textcolor{white}{\textbf{Générateur d'automates}}};
\node[draw=white, fill=couleur3, state, rectangle, rounded corners=4pt] (G) [below =1cm of F] {\textcolor{white}{\textbf{Graphe}}};
\node[draw=white, state, ellipse, fill=couleur2, align=center] (H) [below =1cm of G] {\textcolor{white}{\textbf{Rendu graphique de graphe}} \\ \textcolor{white}{\textbf{(GraphStream)}}};

\node[draw=white, fill=couleur3, state, rectangle, rounded corners=3pt]  (I) [right =5cm of E] {\textcolor{white}{\textbf{Fichiers fournis ou générés}}};
\node[draw=white, state, ellipse, fill=couleur2]  (J) [below=0.5cm of I] {\textcolor{white}{\textbf{Outils à utiliser}}};
\node[draw=white,fill=couleur4, state, rectangle] (K) [below=0.5cm of J] {\textcolor{white}{\textbf{Outil à developper}}};

\draw[<->] (A) -- node [ align=center, below] {Modifier le code \\ source (annotation)}  (B);

\draw[->] (C) -- node [ align=center, above, rotate=90] {Exécution\\ concolique}  (E);

 \path[->]
  (A) edge node{} (C)
  (D) edge node{} (C)
  (F) edge node{} (G)
  (G) edge node{} (H)
  (E) edge node{} (F);
\end{tikzpicture}

\paragraph{}
Nous pouvons extraire de ce graphe un certain nombre de tâches à
effectuer lors du prochain semestre~:

\begin{description}
\item[Annotation automatique] Nous devrons utiliser Soot afin
  d'annoter automatiquement la variable \verb|buffer| contenant les
  informations de l'\gls{APDU}.
\item[JDart] Il faudra tester JDart de manière plus approfondie sur
  différents programmes JavaCard, ainsi que récupérer l'arbre des
  contraintes.
\item[Création d'automates] Il sera question de définir une structure
  de données pour représenter les automates générés par notre
  application, puis d'implémenter les algorithmes en utilisant cette
  structure.
\item[GraphStream] Nous devrons intégrer la bibliothèque GraphStream à
  notre application afin de dessiner l'automate stocké dans la
  structure de données.
\item[Étudier Gephi] Nous devons décider si la bibliothèque Gephi nous
  sera utile, et l'implémenter dans notre programme pour analyser
  l'automate le cas échéant.
\item[Créer une interface graphique] Afin que le programme soit plus
  facilement utilisable, il nous faudra développer une interface
  graphique afin de simplifier l'analyse d'un applet JavaCard par un
  utilisateur.
\end{description}

