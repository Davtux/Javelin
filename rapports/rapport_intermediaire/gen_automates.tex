\chapter{Génération d'un automates}

\paragraph{}
Dans cette partie nous allons présenter l'outil qui nous permettra de créer l'automate d'une applet afin de pouvoir le comparer à l'automate fourni dans les spécifications de l'application.

\section{Soot}

\paragraph{}
À l'origine Soot est framework qui pour but d'optimiser du code. Soot a été produit par le groupe de recherche Sable de l'université de McGill. Concrètement ce framework permet de :

\begin{itemize}
	\item créer des graphes de séquence
	\item créer des graphes de flux de controle
	\item analyser points par points
	\item instrumeter du code
\end{itemize}

\paragraph{}
Soot propose plusieurs représentation intermédiaire du code. Chaque représentation intermédiaire correspond à un niveau d'abstraction.
\begin{itemize}
	\item \textbf{Baf} : est une abstraction du bytecode Java. Il peut être utilisé pour de l'analyse ou de l'optimisation de bytecode.
	\item \textbf{Jimple} : est la représentation principale de Soot. C'est une abstraction à trois addresses, c'est-à-dire qu'une instruction peut contenir deux opérandes. Le jeux d'instruction est naussi réduit (15 instruction contre plus de 200 en bytecode).
	\item \textbf{Shimple} : est proche de Jimple avec un nœud supplémentaire qui permet d'effectué de l'optimisation en supprimant les instruction qui ne sont jamais exécuter.
	\item \textbf{Grimp} : est également basé sur Jimple en supprimant les valeurs intermédiaires.
\end{itemize}


\section{}

\paragraph{}