\chapter{Tâches futures et planification}

\section{Outils utilisés pour la réalisation du projet}
\subsection{Gestionnaire de version}
\paragraph{}
Pour la réalisation de ce projet, il nous a été demandé d'utiliser un gestionnaire de version.

La Direction des Systèmes d'Information ayant mis en place un serveur Git à disposition des étudiants, les responsables du module nous ont demandé de l'utiliser.

\paragraph{}
Un gestionnaire de version nous permettra d'optimiser notre temps de travail en réduisant les regressions de code, ou en permettant de travailler simultanément sur plusieurs facettes de notre application en utilisant le système de branches.

\subsection{Gestionnaire de dépendances}
\paragraph{}
Pour gérer les dépendances de notre projet, nous avons choisi d'utiliser Maven, il s'agit d'un logiciel simplifiant la gestion des dépendances et des versions de chacune d'entre elles.

\paragraph{}
Lors du clonage de notre projet, il suffira donc, théoriquement tout du moins, de laisser Maven configurer les dépendances pour que tout fonctionne simplement.

\subsection{Qualité de code}
\paragraph{}
Pour améliorer ma qualité de notre application finale en simplifiant la maintenance de celle-ci, nous utiliserons SONAR, il s'agit d'une application parcourant le code et vérifiant, entre autres, qu'il ne reste pas d'exeption non interceptée, de commentaires, ou de TODO au sein du code.

\paragraph{}
Afin de simplifier la prise en main de l'application ainsi que sa maintenance, nous utiliserons les fonctionnalités de notre IDE afin de générer au maximum de la JavaDoc. Nous assurerons ainsi la lisibilité de notre code ainsi que sa compréhension.

\section{Plannification de l'avancement du projet au semestre 2}


\section{Quelles tâches à effectuer au semestre prochain~?}

\paragraph{}
Afin d'effectuer les tâches décrites dans le Chapitre~\ref{chap:travail}, il est nécessaire de détailler chacune d'entre elles afin de pouvoir mieux les articuler~:

\begin{itemize}
\item Annotation de variables
  \begin{itemize}
  \item Détecter la variable \verb|buffer| (contenant l'\gls{APDU}) et l'annoter avec \verb|@Symbolic| automatiquement.
  \item Détecter les variables provoquant un changement d'état de l'applet et les annoter automatiquement (par exemple, une variable notant si l'utilisateur est authentifié).
  \end{itemize}
\item JDart
  \begin{itemize}
  \item Générer automatiquement une classe contenant une fonction \verb|main| afin que JDart puisse analyser l'applet.
  \item Générer le fichier de configuration de JDart.
  \item Récupérer la sortie de JDart au format JSON.
  \end{itemize}
\item Création d'un automate
  \begin{itemize}
  \item Définition de la structure de données
  \end{itemize}
\item Représentation graphique
  \begin{itemize}
  \item Adapter notre structure de données à GraphStream
  \end{itemize}
\end{itemize}

\section{Tableau de planification des tâches}

\shorthandoff{:!}

\begin{sidewaystable}[htbp]
  \centering
  \begin{ganttchart}[
    vgrid,hgrid,
    title height=1
    ]{2}{31}
    \gantttitle{Janvier}{6}  \gantttitle{Février}{6} \gantttitle{Mars}{6} \gantttitle{Avril}{6} \gantttitle{Mai}{6} \\
    \gantttitlelist{6,11,16,21,26,31}{1} \gantttitlelist{3,8,13,18,23,28}{1} \gantttitlelist{6,11,16,21,26,31}{1} \gantttitlelist{5,10,15,20,25,30}{1} \gantttitlelist{6,11,16,21,26,31}{1} \\
    \ganttgroup{Group 1}{2}{7} \\
    \ganttbar{Task 1}{2}{2} \\
    \ganttlinkedbar{Task 2}{3}{7} \ganttnewline
    \ganttmilestone{Milestone}{7} \ganttnewline
    \ganttbar{Final Task}{8}{12}
    \ganttlink{elem2}{elem3}
    \ganttlink{elem3}{elem4}
  \end{ganttchart}
\caption{Planning prévisionnel du second semestre}
\end{sidewaystable}
